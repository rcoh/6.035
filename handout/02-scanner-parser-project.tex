\documentstyle[11pt,6035]{article}

\begin{document}

\handout{Handout --- Scanner-Parser Project}{\PARSERASSIGN}

{\bf DUE: \PARSERDUE}

This project consists of two segments: lexical analysis (scanning) and
syntactic analysis (parsing).

\section*{Preliminaries}
In this section we will describe the infrastructure that is provided
for the project.  You are encouraged to use it, but you may also
choose to ignore it and design your infrastructure from scratch.  The
skeleton is located in {\tt /mit/6.035/provided/skeleton}.

\subsection*{Provided Infrastructure}
A skeleton compiler infrastructure has been provided that includes a
typical compiler directory organization, the ANTLR parser and
scanner generator, and an Apache Ant build system. The Java package
name for the sources provided is {\tt decaf}.

The directory structure and the provided files are as follows:

{\scriptsize
\begin{verbatim}
.
|-- bin
|-- build.xml
|-- lib
|   |
|   `-- antlr.jar
`-- src
    |
    |-- decaf
    |   |
    |   |-- Lexer.g
    |   |-- Parser.g
    |   `-- Main.java
    `-- java6035
        |
        `-- tools
            |
            `-- CLI
                |
                `-- CLI.java
\end{verbatim}
}

The {\tt lib} directory includes libraries that are needed during a {\it
run} of your compiler; right now it contains the ANTLR tool. The {\tt
bin} directory contains the tools (executables) needed to {\it compile} your
compiler. However, since ANTLR uses the same binary as a tool and as
a library, and we already included it in the {\it lib} directory,
{\it bin} director is empty. {\tt build.xml} is the ant build file
for the infrastructure. The {\it src} directory contains all your
source code for the compiler. {\tt src/decaf/Parser.g} is a skeleton
parser grammar and {\tt src/decaf/Lexer.g} is a skeleton scanner
grammar. {\tt src/decaf/Main.java} is the skeleton driver and {\tt
src/java6035/tools/CLI/CLI.java} is the command-line interface
library.

To access and build the provided infrastructure, you must add the
following lockers to your file system.
\begin{verbatim}
  add -f java
  add -f git
  add -f eclipse-sdk
\end{verbatim}
The eclipse-sdk locker is optional and it adds the eclipse IDE.  No
changes to your {\tt \$CLASSPATH} variable are required, but due to
an interaction between {\tt sipb} and {\tt java\_v1.6.0\_18}, we
recommend that you set an environment variable {\tt JAVA\_HOME} to
{\tt /afs/athena.mit.edu/software/java\_v1.6.0\_18/}; all the
classes and jar files required for compilation are included by the
Ant build file (see next Section).

\subsection*{Ant Build System}

Please review the Ant build file, {\tt build.xml}, that is
provided. For more information on Ant, please visit:
\begin{verbatim}
  http://ant.apache.org/
\end{verbatim}
To build the system, execute {\tt ant} from the root directory of the system.

The build file includes tasks for running ANTLR on your scanner and parser
grammars, compiling the sources, packaging the compiler into a jar file, and
cleaning the infrastructure.  The default rule is for complete compilation and
jar creation. The build file creates three directories during compilation:
{\tt classes}, {\tt java}, and {\tt dist}.  The {\tt classes} directory
contains all {\tt .class} files created during compilation.  The {\tt java}
directory contains all the generated sources for the compiler (from scanner and
parser grammars). The {\tt dist} directory contains the jar archive file for
the compiler and any other runtime libraries that are necessary for running
the compiler. A clean will delete these three directories. If you are using
revision control, you should not add these directories to your repository.

\subsection*{Getting Started}

A good place to start would be to create a copy of the provided skeleton
and create a git repository to track your changes.  This can be accomplished
with the following commands on athena:
\begin{verbatim}
# initial setup:
add -f java git
cp -r /mit/6.035/provided/skeleton ~/Private/6.035
cd ~/Private/6.035
git init

# save all changes into git repository:
git add .
git commit -m "Initial import"

# build:
ant

# run scanner:
java -jar dist/Compiler.jar -target scan -debug /mit/6.035/provided/scanner/char1

# run parser:
java -jar dist/Compiler.jar -target parse -debug /mit/6.035/provided/parser/legal-01
\end{verbatim}

\section*{Scanner}

Your scanner must be able to identify tokens of the Decaf language,
the simple imperative language we will be compiling in 6.035.  The
language is described in the Decaf Language Handout.  Your scanner should note
illegal characters, missing quotation marks, and other lexical
errors with reasonable error messages.  The scanner should find as
many lexical errors as possible, and should be able to continue
scanning after errors are found.  The scanner should also filter out
comments and whitespace not in string and character literals.

You will not be writing the scanner from scratch. Instead, you will
generate the scanner using ANTLR. This program reads in an input
specification of regular expression-like syntax (grammar) and
creates a Java program to scan the specified language.  More
information on ANTLR (including the manual and examples) can be on
the web at:
\begin{verbatim}
  http://antlr2.org/doc/index.html
  http://www.antlr.org/wiki/display/CS652/CS652+Home
  \end{verbatim}
To get you started, we have provided a template in
\begin{verbatim}
  /mit/6.035/provided/skeleton/src/decaf/Lexer.g
\end{verbatim}
If you chose to work from this skeleton, you must complete the
existing ANTLR rules and add new ones for your scanner.

ANTLR generated scanners throw exceptions when they encounter an
error. Each exception includes a text of a potential error message,
and the provided skeleton driver prints them out. You are free to
use use the messages provided by ANTLR, if you have verified that
they make sense and are specific enough.

ANTLR is invoked by the {\it antlr} task of the Ant build file.  The
generated files, {\tt DecafLexer.java}, etc, are placed in the {\tt
java/decaf} directory by the task.  Note that ANTLR merely generates
a Java source file for a scanner class; it does not compile or even
syntactically check its output. Thus, typos or syntactic errors in
the scanner grammar file will be propagated to the output.

An ANTLR generated scanner produces a string of tokens as its
output. Each token has the following fields:

\begin{tabular}{ll}
   \texttt{type}  & the integer type of the token\\
   \texttt{text}  & the text of the token\\
   \texttt{line}  & the line in which the token appears\\
   \texttt{col}   & the column in which the token appears\\
\end{tabular}

Every distinguishable terminal in your Decaf grammar will have an
automatically generated unique integer associated with it so that
the parser can differentiate them.  These values are created from
your scanner grammar and are stored in the {\tt *TokenTypes.java}
file created by ANTLR.

\subsection*{Parser}

Your parser must be able to correctly parse programs that conform to
the grammar of the Decaf language.  Any program that does not conform
to the language grammar must be flagged with at least one error
message.

As mentioned, we will be using the ANTLR LL(k) parser generator, same
tool as for the Scanner. You will need to transform the reference grammar
in Decaf Language Handout into a grammar expressed in the ANTLR grammar
syntax. Be careful with checking your spelling, as ANTLR does match rule
uses to declarations.

Your parser does not have to, and should not, recognize
context-sensitive errors \eg using an integer identifier where an
array identifier is expected.  Such errors will be detected by the
static semantic checker.  {\it Do not} try to detect
context-sensitive errors in your parser.  However, you might need to
create syntactic actions in your parser to check for some
context-free errors.

You might want to look at Section 3 in the ``Tiger'' book, or
Sections 4.3 and 4.8 in the Dragon book for tips on getting rid of
shift/reduce and reduce/reduce conflicts from your grammar.  You can
tell ANTLR to print out the parse states (useful for resolving
conflicts) by adding {\tt trace="yes"} to the ANTLR target (please
see the build file).

\newpage

\subsection*{What to Hand In}

Projects should be submitted electronically through stellar.  You should
submit a gzipped tar file named {\tt LASTNAME-parser.tar.gz}, where LASTNAME
is replaced with your last name.  The contents of this file should follow
the following structure:

{\scriptsize
\begin{verbatim}
LASTNAME-parser.tar.gz
|
`-- LASTNAME-parser
    |
    |-- code
    |   |
    |   ...                (full source code, can build by running `ant`)
    |
    |-- doc 
    |   |
    |   ...                (write-up, described in project overview handout)
    |   
    `-- dist 
        |
        `-- Compiler.jar   (compiled output, for automated testing)
\end{verbatim}
}

It is expected that most (or all) students will use Java for their project, and
thus should adhere to the above format.  If you elect to use a language other
than Java, you must provide a wrapper script ./run.sh (runnable with bash)
in the main directory that calls your code and accepts the same arguments
described in the project overview handout.  Any additional dependencies
and instructions to build your code should be included in the documentation.

\subsubsection*{Scanner output format}

When {\tt -target scan} is specified, the output of your compiler
should be a scanned listing of the program with one row for each
token in the input. Each line will contain the following
information: the line number (starting at 1) on which the token
appears, the type of the token (if applicable), and the token's
text. Please print only the following strings as token types (as
applicable): {\tt CHARLITERAL}, {\tt INTLITERAL}, {\tt
BOOLEANLITERAL}, {\tt STRINGLITERAL} and {\tt IDENTIFIER}.

For {\tt STRINGLITERAL} and {\tt CHARLITERAL}, the text of the token
should be the text, as appears in the original program, including
the quotes and any escaped characters.

Each error message should be printed on its own line, {\it before}
the erroneous token, if any.  Such messages should include the file
name, line and column number on which the erroneous token appears.

Here is an example table corresponding to {\tt print("Hello,
  World!");}:

\begin{center}
\begin{tabular}{l}
\texttt{1 IDENTIFIER print} \\
\texttt{1 (}\\
\texttt{1 STRINGLITERAL "Hello, World!"} \\
\texttt{1 )} \\
\texttt{1 ;}
\end{tabular}
\end{center}

You are given both a set of test files on which to test your scanner
and the expected output for these files (see next Section).  {\bf
The TA requests that the output of your scanner matches the provided
output exactly on all files without errors} (successful exit status
of the {\tt diff} command). The TA would like to automate the
grading process as much as possible.

\subsubsection*{Parser output format}
When {\tt -target parse} is specified, any syntactically incorrect
program should be flagged with at least one error message, and the
program should exit with a non-zero value (see {\tt System.exit()}).
Multiple error messages may be printed for programs with multiple
syntax errors that are amenable to error recovery. Given a
syntactically valid program, your parser should produce no output,
and exit with the value zero (success). The exact format for parse
error messages is not stipulated.

\section*{Provided Test Cases and Expected Output}

The provided test cases for scanning and parsing can be found in:

\begin{verbatim}
/mit/6.035/provided/scanner
/mit/6.035/provided/parser
\end{verbatim}

The expected output for each scanner test can be found in:
\begin{verbatim}
/mit/6.035/provided/scanner/output
\end{verbatim}

We will test your scanner/parser on these and a set of hidden tests.
You will receive points depending on how many of these tests are
passed successfully.  In order to receive full credit, we also expect
you to complete the written portion of the project as described in the
Project Overview Handout.

\newpage
\appendix
\section{Why we defer integer range checking until the next project}

When considering the problem of checking the legality of the input
program, there is no fundamental separation between the
responsibilities of the scanner, the parser and the semantic checker.
Often, the compiler designer has the choice of checking a certain
constraint in a particular phase, or even of dividing the checking
across multiple phases.  However, for pragmatic reasons, we have had
to divide the complete scan/parse/check unit into two parts simply to
fit the course schedule better.

As a result, we will have to mandate certain details about your
implementations that are not necessarily issues of correctness.  For
example, one cannot completely check whether integer literals are
within range without constructing a parse tree.

Consider the input:

\begin{verbatim}
    x+-2147483648
\end{verbatim}

This corresponds to a parse tree of:

\begin{verbatim}
    +
   / \
  x   -
      |
  INT(2147483648)
\end{verbatim}

We cannot confirm in the scanner that the integer literal -2147483648
is within range, since it is not a single token.  Nor can we do this
within the parser, since at this stage we are not constructing an
abstract syntax tree.  Only in the semantic checking phase, when we
have an AST, are we able to perform this check, since it requires the
unary minus operator to modify its argument if it is an integer
literal, as follows:

\begin{verbatim}
    +                         +
   / \                       / \
  x   -           ---->     x   INT(-2147483648)
      |
  INT(2147483648)
\end{verbatim}

Of course, if the integer token was clearly out of range
(e.g. 99999999999) the scanner could have rejected it, but this check
is not required since the semantic phase will need to perform it later
anyway.

Therefore, rather than do some checking earlier and some later, we
have decided that ALL integer range checking must be deferred until
the semantic phase.  So, your scanner/parser must not try to interpret
the strings of decimal or hex digits in an integer token; the token
must simply retain the string until the semantic phase.

When printing out the token table from your scanner, do not print the
value of an {\tt INTLITERAL} token in decimal.  Print it exactly as
it appears in the source program, whether decimal or hex.

%% \section{Eclipse}

%% This section includes some brief instructions that will get you
%% started with the Eclipse IDE. Eclipse is located in the {\tt
%%   eclipse-sdk} locker and can be run by issuing {\tt eclipse}.

%% To create a project from a local CVS repository on Athena,

%% To create a project from a remote CVS repository, Select ``File
%% $\Rightarrow$ New $\Rightarrow$ Project'' and select ``CVS
%% $\Rightarrow$ Checkout Projects from CVS'' and click ``Next''.  At the
%% next screen, enter {\tt ant.mit.edu} for ``Host:'', the absolute path
%% for you CVS repository for ``Repository path:, your username/password
%% for ``User:'' and ``Password'', and ``extssh'' for ``Connection
%% type:''.  Click ''Next''.

%% To use your build file in Eclipse, create a new Ant Builder for the
%% project under the ``Project $\Rightarrow$ Properties'' menu,
%% ``Builders'' tab. Select ``New...'' and create a new builder of type
%% ``Ant Build''.  In the ``Main'' tab, select your build file
%% (build.xml) and set the base directory to the base directory of your
%% project.  You will want to enable resource refreshing under the
%% ``Refresh'' tab. In the ``Targets'', set target of each operation as
%% you see fit given the build file.

\end{document}
