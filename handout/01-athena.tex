\documentstyle[11pt,6035]{article}

\begin{document}

\handout{Handout --- Athena, Tools}{\CLASSONE}

This document describes what you will need to know about Athena and
Java tools for 6.035.

\subsection*{Athena clusters}

If you don't have an account on Athena, you should register for one
immediately.  Information can be found at {\tt http://ist.mit.edu/services/athena}

You can work in any of the public Athena clusters.  Type {\tt cview} to see a
list of clusters and available machines.

\subsection*{Communication}

We'll make course announcements via electronic mail.  If you don't
receive a message welcoming you to the class mailing list within a
week, tell the TA immediately.

We'll answer questions via email.  You can mail your TA directly
or use {\tt 6.035-staff@mit.edu} to reach the entire staff.

\subsection*{Finding course files}

Handouts will be available on the course web site.  Project
directories, examples, and 6.035 programs are stored in {\tt
/mit/6.035}.  The Java compiler, library, debugger, and associated
programs and documentation are stored in the Java locker {\tt
/mit/java}.

You'll probably want to add these lines to your {\tt .environment} file:
\begin{verbatim}
  add -f java
  add -f git
\end{verbatim}

These commands attach the lockers and update your execution path
to include the course software.

If you need to use different versions of java for other classes you
can use the -ver switch.
\begin{verbatim}
    java -ver 1.6.0_23  .....
    javac -ver 1.6.0_23 .....
    etc.
\end{verbatim} 
Note that the -ver must come before any other command line arguments.

\subsection*{Working with Groups}

The first project should be done individually.  When the second project is
assigned, the class will be partitioned into groups of 3 or 4 students.

Each group will be given a group locker that can be used to work on
the project.  There should be enough space in these lockers that you
won't have any problems.  Only group members and 6.035 staff will
be able to access each group locker.

\subsection*{Java Compiler}

We'll be using Sun's JDK 1.6.0. It is available on the Sun
and Linux Athena platforms.  You can get a free version of the JDK
from Sun's web site for other platforms (Windows, Mac).  However, the
only officially supported platforms for this class are Sun and Linux,
so we may not be able to help you if you run into problems with other
platforms.  Since Java is platform independent, you can compile your
final bytecodes on any platform.

Below we describe basic operation of JDK 1.6.0.  For detailed
information on JDK and Java API 1.6.0, consult {\tt
http://java.sun.com/}.

\subsubsection*{Running JDK 1.6.0}

Compile the source file(s) using the Java compiler. Use the {\tt -g}
flag to create debuggable bytecodes.

\begin{verbatim}
  % javac dummy1.java dummy2.java
\end{verbatim}

If compilation succeeds, the compiler creates a {\tt .class} file
(named {\it ClassName}{\tt.class}) for each public class defined.  If
compilation fails, the compiler lists compilation errors.

You must have no more than one public class defined in each of the
source files, and the filename must be {\em exactly} the same as the
class name.  You can define several private classes in one file,
however the compiler will still generate a separate {\tt .class} file
for each class.

In 6.035 we'll be writing Java applications (not applets).  Each Java
application must have a public class that contains a {\tt public static
void main(String[] args)} method. To run the program simply type

\begin{verbatim}
  % java MyProgram arg1 arg2 arg3
\end{verbatim}

where MyProgram is the name of the class with the {\tt main} method,
and {\tt arg1, arg2, arg3} are the command line arguments. These
arguments are passed to the program in the {\tt args} parameter to the
{\tt main} method, and can be accessed as {\tt args[0], args[1]}, etc.

\subsubsection*{Java Debugger}

If you use the {\tt -g} flag when you compile your source code, you
will be able to debug your program using the JDK debugger. Start the
debugger using:

\begin{verbatim}
  % java -debug MyClass
\end{verbatim}

Once inside the debugger, {\tt help} will list all available commands.
Here is a very limited list of the most useful:
\begin{itemize}
\item {\tt run $<$class$>$ [args]} --- Start execution of a loaded Java class
\item {\tt print $<$id$>$ [id(s)]} --- Print object or field 
\item {\tt stop in $<$class id$>$.$<$method$>$} --- Set a breakpoint in a $<$method$>$
\item {\tt cont} --- Continue execution to the next break point.
\item {\tt locals} --- Print all local variables in current stack frame
\item {\tt help} --- Displays the list of recognized commands with descriptions 
\end{itemize}

\subsection*{Ant}

You are required to use {\tt Apache Ant} to build your projects, and
to provide a {\tt build.xml} the TA can use to create bytecodes from
your source files.  Ant is a java-based build tool that strives to be
platform independent.  Ant is much like the familiar {\tt Make} tools
in that it resolves the dependencies necessary to perform a task.
However, Ant is not shell-based.  Project configuration files are
written in hierarchical xml and Ant is extended by implementing java
classes.  Please visit {\tt http://ant.apache.org/} for more
information on Ant.  The handout describing the Scanner and Parser
phase of the project will include a more detailed description of the
build system.

% Unfortunately,
% the current configuration of Ant and Java on Athena is somewhat
% inconsistant, and by default Ant uses JDK 1.6. This means that when
% you run your program using {\tt java -jar Compiler.jar ...},
% you may receive an error message: {\tt Bad version number}. To fix
% this, you may want to set {\tt JAVA\_HOME} environment variable to
% {\tt /afs/athena.mit.edu/software/java\_v1.5.0\_06/}. This will let
% Ant know where to find the correct version of Java.

\subsection*{Revision Control}

You should use revision control on your projects to help manage changes to
your source code base. You'll be writing a lot of code for 6.035, and for
many of you this will be the first time you'll be working on a project of
sufficient complexity to require source control.

We recommend {\tt git} for revision control.  Although the choice of other
system (such as svn, bzr, or mercurial) is left up to each group.

More information on git can be found here: {\tt
http://git-scm.com/documentation}

It can be found in the git locker on athena: {\tt add -f git}

% There is an online manual available at {\tt
% http://ximbiot.com/cvs/manual/}, as well as lots of information on
% the man page ({\tt man cvs}).  You'll need to
% understand the concepts of creating a repository ({\tt cvs init}),
% checking out a working copy of your source tree ({\tt cvs co}),
% checking in changes ({\tt cvs commit}), getting diffs and logs ({\tt
% cvs diff} and {\tt cvs log}), and updating your working directory
% ({\tt cvs update}).  You may also find tagging and branching useful.
% CVS also supports remote repositories (using SSH) that make it much
% easier for group members to work from home.

% CVS is in the GNU locker on athena: {\tt add gnu}

% Note: If you would like to use remote CVS, the TA has found that {\tt
%   ant.mit.edu} is currently the most reliable server for your
% connection.

\end{document}
